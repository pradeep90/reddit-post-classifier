% This is "sig-alternate.tex" V2.1 April 2013
% This file should be compiled with V2.5 of "sig-alternate.cls" May 2012
%
% This example file demonstrates the use of the 'sig-alternate.cls'
% V2.5 LaTeX2e document class file. It is for those submitting
% articles to ACM Conference Proceedings WHO DO NOT WISH TO
% STRICTLY ADHERE TO THE SIGS (PUBS-BOARD-ENDORSED) STYLE.
% The 'sig-alternate.cls' file will produce a similar-looking,
% albeit, 'tighter' paper resulting in, invariably, fewer pages.
%
% ----------------------------------------------------------------------------------------------------------------
% This .tex file (and associated .cls V2.5) produces:
%       1) The Permission Statement
%       2) The Conference (location) Info information
%       3) The Copyright Line with ACM data
%       4) NO page numbers
%
% as against the acm_proc_article-sp.cls file which
% DOES NOT produce 1) thru' 3) above.
%
% Using 'sig-alternate.cls' you have control, however, from within
% the source .tex file, over both the CopyrightYear
% (defaulted to 200X) and the ACM Copyright Data
% (defaulted to X-XXXXX-XX-X/XX/XX).
% e.g.
% \CopyrightYear{2007} will cause 2007 to appear in the copyright line.
% \crdata{0-12345-67-8/90/12} will cause 0-12345-67-8/90/12 to appear in the copyright line.
%
% ---------------------------------------------------------------------------------------------------------------
% This .tex source is an example which *does* use
% the .bib file (from which the .bbl file % is produced).
% REMEMBER HOWEVER: After having produced the .bbl file,
% and prior to final submission, you *NEED* to 'insert'
% your .bbl file into your source .tex file so as to provide
% ONE 'self-contained' source file.
%
% ================= IF YOU HAVE QUESTIONS =======================
% Questions regarding the SIGS styles, SIGS policies and
% procedures, Conferences etc. should be sent to
% Adrienne Griscti (griscti@acm.org)
%
% Technical questions _only_ to
% Gerald Murray (murray@hq.acm.org)
% ===============================================================
%
% For tracking purposes - this is V2.0 - May 2012

\documentclass{sig-alternate-05-2015}

\usepackage{hyperref}

\begin{document}

% Copyright
\setcopyright{acmcopyright}
%\setcopyright{acmlicensed}
%\setcopyright{rightsretained}
%\setcopyright{usgov}
%\setcopyright{usgovmixed}
%\setcopyright{cagov}
%\setcopyright{cagovmixed}


% DOI
\doi{10.475/123_4}

% ISBN
\isbn{123-4567-24-567/08/06}

%Conference
%% \conferenceinfo{PLDI '13}{June 16--19, 2013, Seattle, WA, USA}

\acmPrice{\$15.00}

%
% --- Author Metadata here ---
%% \conferenceinfo{WOODSTOCK}{'97 El Paso, Texas USA}
%\CopyrightYear{2007} % Allows default copyright year (20XX) to be over-ridden - IF NEED BE.
%\crdata{0-12345-67-8/90/01}  % Allows default copyright data (0-89791-88-6/97/05) to be over-ridden - IF NEED BE.
% --- End of Author Metadata ---

\title{Classification of Reddit Text Posts}
%% \subtitle{[Extended Abstract]
%% \titlenote{A full version of this paper is available as
%% \textit{Author's Guide to Preparing ACM SIG Proceedings Using
%% \LaTeX$2_\epsilon$\ and BibTeX} at
%% \texttt{www.acm.org/eaddress.htm}}}
%
% You need the command \numberofauthors to handle the 'placement
% and alignment' of the authors beneath the title.
%
% For aesthetic reasons, we recommend 'three authors at a time'
% i.e. three 'name/affiliation blocks' be placed beneath the title.
%
% NOTE: You are NOT restricted in how many 'rows' of
% "name/affiliations" may appear. We just ask that you restrict
% the number of 'columns' to three.
%
% Because of the available 'opening page real-estate'
% we ask you to refrain from putting more than six authors
% (two rows with three columns) beneath the article title.
% More than six makes the first-page appear very cluttered indeed.
%
% Use the \alignauthor commands to handle the names
% and affiliations for an 'aesthetic maximum' of six authors.
% Add names, affiliations, addresses for
% the seventh etc. author(s) as the argument for the
% \additionalauthors command.
% These 'additional authors' will be output/set for you
% without further effort on your part as the last section in
% the body of your article BEFORE References or any Appendices.

\numberofauthors{2} %  in this sample file, there are a *total*
% of EIGHT authors. SIX appear on the 'first-page' (for formatting
% reasons) and the remaining two appear in the \additionalauthors section.
%
\author{
% You can go ahead and credit any number of authors here,
% e.g. one 'row of three' or two rows (consisting of one row of three
% and a second row of one, two or three).
%
% The command \alignauthor (no curly braces needed) should
% precede each author name, affiliation/snail-mail address and
% e-mail address. Additionally, tag each line of
% affiliation/address with \affaddr, and tag the
% e-mail address with \email.
%
% 1st. author
\alignauthor
Pradeep Kumar Srinivasan\\
\affaddr{Purdue University}\\
\email{sriniv68@purdue.edu}
% 2nd. author
\alignauthor
Mohammad Haseeb\\
\affaddr{Purdue University}\\
\email{mhaseeb@purdue.edu}
}
% There's nothing stopping you putting the seventh, eighth, etc.
% author on the opening page (as the 'third row') but we ask,
% for aesthetic reasons that you place these 'additional authors'
% in the \additional authors block, viz.
\date{\today}
% Just remember to make sure that the TOTAL number of authors
% is the number that will appear on the first page PLUS the
% number that will appear in the \additionalauthors section.

\maketitle
\begin{abstract}

We plan to work on a large text classification problem, specifically the Reddit Self-Post Classification Task (RSPCT). RSPCT is a dataset with around 1000 classes (``subreddits'') with around 1000 examples per class, which is unique because most text classification datasets have sparse labels. We plan to use two traditional machine learning algorithms and one deep learning algorithm to learn to predict the class given the title and body of a post.

\end{abstract}


% %
% % The code below should be generated by the tool at
% % http://dl.acm.org/ccs.cfm
% % Please copy and paste the code instead of the example below.
% %
% \begin{CCSXML}
% <ccs2012>
%  <concept>
%   <concept_id>10010520.10010553.10010562</concept_id>
%   <concept_desc>Computer systems organization~Embedded systems</concept_desc>
%   <concept_significance>500</concept_significance>
%  </concept>
%  <concept>
%   <concept_id>10010520.10010575.10010755</concept_id>
%   <concept_desc>Computer systems organization~Redundancy</concept_desc>
%   <concept_significance>300</concept_significance>
%  </concept>
%  <concept>
%   <concept_id>10010520.10010553.10010554</concept_id>
%   <concept_desc>Computer systems organization~Robotics</concept_desc>
%   <concept_significance>100</concept_significance>
%  </concept>
%  <concept>
%   <concept_id>10003033.10003083.10003095</concept_id>
%   <concept_desc>Networks~Network reliability</concept_desc>
%   <concept_significance>100</concept_significance>
%  </concept>
% </ccs2012>
% \end{CCSXML}

% \ccsdesc[500]{Computer systems organization~Embedded systems}
% \ccsdesc[300]{Computer systems organization~Redundancy}
% \ccsdesc{Computer systems organization~Robotics}
% \ccsdesc[100]{Networks~Network reliability}


%
% End generated code
%

%
%  Use this command to print the description
%
\printccsdesc

% We no longer use \terms command
%\terms{Theory}

\keywords{NLP; large text classification; multi-class classification}

\section{Introduction}

% A 2-page maximum document describing:
% + The composition of your team
% + The topic you plan to work on for the final project; in particular, please explain why you
% believe the topic you've chosen is an interesting and innovative topic.
% + Your plan of activities to conduct in your project (e.g., literature survey, data collection
% and exploration, algorithm design and implementation, evaluation, etc.)
% + Your plan to evaluate the outcome of your project (e.g., what do you expect to achieve
% through your project? How will you measure whether your project achieve the intended
% goals?)
% + Your project timeline (e.g., how much time will you spend on each of the activities you
% plan to conduct for your project?)

Text classification with few classes, such as in sentiment analysis, has been well-studied \cite{sentiment-analysis} with state-of-the-art techniques like LSTM. However, those techniques do not always work as well in scenarios with many classes \cite{many-classes}. Another issue with training on many-class datasets is that they have a very large number of labels (such as WikiLSHTC-325K dataset \cite{partalas2015lshtc}, which has 325K labels) and are sparse - most labels in the above dataset have less than 100 examples \cite{jonesreddit}.

Reddit Self-Post Classification Task (RSPCT) is a text corpus containing self-posts (i.e., text posts) from Reddit. RSPCT was collected to help spur research on models that could tackle a large number of classes by ensuring a large population of examples for each class (currently, around 1000 examples for each class) and a large number of classes (around 1000). The aim is to allow for a situation comparable to that in the computer vision community, which was helped by the famous ILSRVC competition (ImageNet \cite{imagenet}) with 1000 classes and around 1400 examples per class. This potential is what made us find this project interesting.

We think this is a useful problem to solve because it has not been adequately studied, to the best of our knowledge, and because this could pave the way to future work on such many-class datasets. We hope to try both traditional and deep learning models so that we get a good sense of their relative performance on such datasets.

\section{Plans for Exploratory Analysis}

In order to explore the data and get a feel of it, we intend on using different methods. We will be plotting several histograms to get to know the median length of the subreddits in terms of both words and characters in the subreddits. We hope to get a deeper understanding of the lengths of the subreddits in general using this method. Because the dataset contains thousands of different words and each word is a possible feature, we will be using t-Distributed Stochastic Neighbor Embedding (t-SNE)\cite{tsne} for visualizing this high-dimensional dataset.

As part of pre-processing, we will remove stopwords and any other undesirable tokens from the dataset.

We will compute sentiment ratings for different posts to see if there are any meaningful correlations across subreddits. If so, we will add that as a new feature to our dataset. We want to use TextBlob\cite{textblob}, or a better alternative if we find one, to get the median sentiment value of each subreddit. We will then be able to use this to determine which subreddit a new post can belong to given its sentiment value. Another potentially useful feature might be the readability level. We plan to compute the readability of the text \cite{readability} and check for meaningful correlations.

%% Similarly, we plan on using a text readability tool to determine the readability score for a certain subreddit and use the text readability of a new post to determine its label.

\section{Algorithms we aim to Implement}

We aim to implement two traditional classification algorithms and one deep learning algorithm.

First, we plan to use Naive Bayes classification because of its simplicity and past track record in multi-class text classification \cite{rennie2001:improving}. This will provide us with a good baseline against which to test the other models. Just in case, we will also use a trivial classifier (one that predicts classes at random) as another baseline.

Second, we plan to use logistic regression with a bag-of-words model based on word n-grams and character n-grams \cite{logistic-regression:bag-of-words}. Since we are not sure if logistic regression or SVM will be better suited to this problem, we plan to answer this question using our literature review. If prior research suggests SVM performs better on large text classification, we may switch to it.

Lastly, we plan to use Convolutional Neural Networks (CNN) to test whether deep learning models can outperform the traditional models on text classification problems with many classes. Again, we will check if the existing literature recommends other deep learning algorithms for text classification and may switch to something like RNN.

\section{Proposed Experiments}

On top of the previous three algorithms, we will experiment with adding a new feature - the sentiment of the text (which we will compute using a pre-trained, off-the-shelf model \cite{textblob}) - to see if it improves performance.

\section{Evaluation of our Project}

The key metric for the models will be the Precision-at-K metric \cite{krizhevsky2012imagenet}, since this is a problem with a very large number of classes and we don't expect the correct label to be predicted as the top option. We will measure the precision on the top-1, top-3, and top-5 labels. As mentioned before, we will also experiment with adding a new feature for sentiment rating of the text to see if that improves performance. To understand the impact of the title on classification, we will measure performance by training and testing on just the titles of the posts, just the bodies of the posts, and then with both title and body.

%% Finally, we will do a little qualitative evaluation by contrasting one or two posts for which our model confidently predicts the wrong category.

In addition, for the deep learning model, we will plot the loss and accuracy curves vs number of epochs. For the traditional models, we will plot the learning curve vs training set size.

\section{Timeline}

Week 1: We will do exploratory data analysis and study the literature to learn the strengths and weaknesses of past approaches towards large text classification.

Week 2 and Week 3: We will implement and train the different models.

Week 4: We will tune our hyperparameters to get the best performance on the validation sets. We hope to finetune variables like learning rate, batch size, and number of epochs.

Week 5: We will spend this week writing our report and presentation (and finishing any work that is left from the previous weeks).

\section{Conclusions}

In conclusion, we think this would be a interesting attack on the problem of large text classification with many classes. The fact that we are covering both traditional and deep learning models will help us judge which models are likely to work well on these kinds of datasets. The models we train could find use in judging posts that are off-topic to a particular subreddit and might suggest more suitable subreddits. With more data, we could transfer this approach to classifying news articles by subgenres or blog posts by topic.

%\end{document}  % This is where a 'short' article might terminate

%
% The following two commands are all you need in the
% initial runs of your .tex file to
% produce the bibliography for the citations in your paper.
\bibliographystyle{abbrv}
\bibliography{sigproc}  % sigproc.bib is the name of the Bibliography in this case
% You must have a proper ".bib" file
%  and remember to run:
% latex bibtex latex latex
% to resolve all references
%
% ACM needs 'a single self-contained file'!
%% \subsection{References}

\end{document}

%%% Local Variables:
%%% mode: latex
%%% TeX-master: t
%%% End:
